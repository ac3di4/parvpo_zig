\subsection{Векторизация операций}

Zig поддерживает паралельную работу с векторами булей, целых и дробных чисел и указателей при помощи SIMD инструкций.
Они создаются при помощи функции \textit{@Vector} и поддерживают следующие операции:
\begin{itemize}
    \item Арифметические: +, -, /, *, @sqrt, @log и пр.;
    \item Побитовые: >>, <<, \&, |, ~, и пр.;
    \item Сравнения: >, <, = и пр.
\end{itemize}

Так же есть функции для работы со всем вектором:
\begin{itemize}
    \item @splat -- делает вектор из одного скаляра (повтор n раз);
    \item @reduce -- использует оператор для получения скаляра из вектора. Доступны арифметические и логические операторы, а так же взятие максимума и минимума;
    \item @shuffle -- собирает один вектор из нескольких с помощью специальной маски;
    \item @select -- объединяет два вектора в один с помощью предиката (вектор булей).
\end{itemize}

Важно отметить, что если целевая платформа не поддерживает SIMD инструкции, то операции над каждым элеметом вектора выполняются последовательно.

Пример использования векторов:
\begin{lstlisting}
const a = @Vector(4, i32){1, 2, 3, 4};
const b = @Vector(4, i32){5, 6, 7, 8};
const c = a + b; // c = [6, 8, 10, 12]
\end{lstlisting}