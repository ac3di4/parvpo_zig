\subsection{Ключевое слово \textit{const}}

В языке программирования Zig ключевое слово конст используется для объявления новых переменных следующим образом:
\begin{lstlisting}
const a = 0;
const b: i32 = 1;
\end{lstlisting}

Тип константы можно не задавать, однако тип зачастую требуется при инициализации новой изменяемой переменной:
\begin{lstlisting}
var a = 0;      // ERROR
var b: i32 = 1; // ОК
\end{lstlisting}

При этом, если изменяемая переменная не изменяется, то компилятор так же выдаст ошибку:
\begin{lstlisting}
var a: usize = 0;
var l = 1000000000; // error: variable must be const or comptime
for (0 .. l) |b|
    a = function(b);
std.debug.print("{}\n", .{a});
\end{lstlisting}

Однако, стоит отметить, что это требование можно обойти с помощью вызова функции. Например:
\begin{lstlisting}
fn num() callconv(.Inline) usize {
    return 1000000000;
}

// some code

var l = num(); // OK
comptime l = num(); // OK
var d = std.math.pow(usize, 10, 9); // OK
\end{lstlisting}

Все же я полагаю это временной недоработкой языка и так как согласно идеологии и инструметам Zig как можно больше вычислений следует оставлять компилятору, то полагаю, что сравнивать исполнение с \textit{const} и без него не имеет смысла. Потому что \textit{const} следует использовать по возможности всегда.
