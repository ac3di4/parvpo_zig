\section{Введение}

\subsection{История и цель создания}

Язык программирования Zig был создан в 2015 году американским программистом Эндрю Келли (Andrew Kelley) с целью предоставить разработчикам гибкий и мощный инструмент для создания высокопроизводительных системных приложений, который сочетает в себе простоту и эффективность.

Не смотря на свою "свежесть", он уже может быть использован для написания программного обеспечения. Одним из примеров служит bun, пакетный менеджер JavaScript, написанный на Zig. Согласно заявлениям разработчика это помогло уменьшить количество потребляемой памяти и ускорить работу программы по сравнению с остальными пакетными менеджерами.

Так же автор Zig помимо полноценной работы над языком участвует в конференциях. Одной из них является GOTO 2022.

\subsection{Задачи, решаемые языком}

Язык программирования Zig подходит для разработки высокопроизводительных системных приложений, таких как операционные системы, драйверы устройств, сетевые приложения, компиляторы, библиотеки и другие схожие системные проекты. Язык сочетает в себе простоту и эффективность, что является несомненным преимуществом.

Однако, Zig плохо подходит для разработки приложений, которые требуют быстрого прототипирования, так как язык является системным и тот уровень работы с "железом", который он предоставляет может не понадобиться или даже мешать.

Также стоит учитывать, что Zig является относительно новым языком программирования, и на данный момент у него ограниченное количество библиотек и инструментов. Это частично исправляется тем, что Zig может использовать библиотеки C (импортируются в одну строчку) и тем, что сообщество разработчиков языка активно работает над их созданием.

\subsection{Производительность в тестах}

Результаты тестов Zig неоднозначны. С одной стороны среди трех тестов в которых он присутствует (1.1, 1.2, 2.1) только в тесте 1.1 он занимает второе место -- в остальных первое. С другой стороны, сравнивать производительность компилируемого и скриптового языков не корректно (Python, JavaScript и пр.), так как у первого есть преимущество. В то же время языки, с которыми чаще всего сравнивают Zig (C, C++, Rust) в тестах отсутствуют.

Единственный вывод, который можно сделать на основе этих данный это то, что Zig не является настолько неоптимизированным, что проигрывает интерпретируемым языкам.